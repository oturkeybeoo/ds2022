{\rtf1\ansi\ansicpg1252\cocoartf2580
\cocoatextscaling0\cocoaplatform0{\fonttbl\f0\fswiss\fcharset0 Helvetica;}
{\colortbl;\red255\green255\blue255;}
{\*\expandedcolortbl;;}
\paperw11900\paperh16840\margl1440\margr1440\vieww11520\viewh8400\viewkind0
\pard\tx566\tx1133\tx1700\tx2267\tx2834\tx3401\tx3968\tx4535\tx5102\tx5669\tx6236\tx6803\pardirnatural\partightenfactor0

\f0\fs24 \cf0 \\documentclass[13pt]\{article\}\
\\renewcommand\{\\baselinestretch\}\{1.0\}\
\\usepackage[utf8]\{vietnam\}\
\\usepackage[a4paper, total=\{6in, 8in\}]\{geometry\}\
\\usepackage[vietnamese,english]\{babel\}\
\\usepackage\{hyperref\}\
\\usepackage\{mathtools\}\
\\usepackage\{amssymb\}\
\\usepackage\{indentfirst\}\
\\usepackage\{graphicx\}\
\\usepackage\{minted\}\
\\usepackage\{ragged2e\}\
\\usepackage[nottoc]\{tocbibind\}\
\
\\hypersetup\{\
    colorlinks=true,\
    linkcolor=blue,\
    citecolor=blue,\
    urlcolor=blue,\
\}\
\\begin\{document\}\
\\begin\{titlepage\}\
    \\begin\{center\}\
        \\vspace*\{1.8cm\}\
        \\Large\
        Distributed System Labwork 1\\\\\
        \\Large\
        \\vspace\{0.5cm\}\
        \\begin\{center\}\
            \\includegraphics[scale=1.0]\{logo USTH-01.PNG\}\
        \\end\{center\}  \
        \\vspace\{0.5cm\}\
            Group 6 - ICT\\\\\
        \\vspace\{0.5cm\}\
            University of Science and Technology of Hanoi\\\\\
        \\vspace\{0.5cm\}\
            January, 2022\
        \\vfill\
          \
   \\end\{center\}\
\\end\{titlepage\}\
\
\\newpage\
\\tableofcontents\
\\newpage\
\
\\section\{Introduction\}\
\\subsection\{Overview\}\
\\noindent%\
Based on the given chat system, we try to develop a file transfer via TCP/IP in CLI in this labwork.\
\
\\subsection\{Protocol\}\
\
\\begin\{figure\}[h]\
    \\centering\
    \\includegraphics[scale=0.15]\{protocol-01.png\}\
    \\caption\{Protocol diagram\}\
    \\label\{fig:protocol\}\
\\end\{figure\}\
\
\\subsection\{System organization\}\
\\noindent%\
The server establishes a unique port, which in our case is 12345, which was provided by our teacher. It accepts one argument, the IP address, from the client CLI. A single client connects to a single server. The client sends data through a buffer, which is a character array. The buffer's maximum size is 1024 bytes. The server will write the buffer to the file after receiving it. After reaching the end-of-file, the client will notify the server that there is nothing remaining. Both the server and the client will then shutdown.\
\
\\begin\{figure\}[H]\
    \\centering\
    \\includegraphics[scale=0.25]\{system-01.png\}\
    \\caption\{System organization\}\
\\end\{figure\}\
\
\\subsection\{Implementation\}\
\\noindent%\
From the client, we type gcc client.c -o client. Then type ./client localhost\
\\noindent%\
We have implemented the client side:\
\
\\begin\{minted\}\{c\}\
#include <unistd.h>\
#include <stdio.h>\
#include <stdlib.h>\
#include <string.h>\
#include <sys/types.h>\
#include <sys/socket.h>\
#include <netdb.h>\
\
int main(int argc, char* argv[]) \{\
    int so;\
    char s[100];\
    struct sockaddr_in ad;\
\
    socklen_t ad_length = sizeof(ad);\
    struct hostent *hep;\
\
    // create socket\
    int serv = socket(AF_INET, SOCK_STREAM, 0);\
\
    // init address\
    hep = gethostbyname(argv[1]);\
    memset(&ad, 0, sizeof(ad));\
    ad.sin_family = AF_INET;\
    ad.sin_addr = *(struct in_addr*)hep->h_addr_list[0];\
    ad.sin_port = htons(12345);\
\
    // connect to server\
    connect(serv, (struct sockaddr*)&ad, ad_length);\
    \
    memset(&s, 0, 100);\
    FILE* file;\
    file = fopen("send_file.txt", "r");\
    if (file == NULL) \{\
    	printf("The file is null");\
    \} else \{\
    	printf("Read file successfully\\n");\
    \}\
    \
    char buffer[1024] = \{0\};\
    while (fgets(buffer, sizeof(buffer), file) != NULL) \{\
    	int i = send(serv, buffer, sizeof(buffer), 0);\
    	if (i == -1) \{\
    		printf("Send data fail");\
    	\}\
    	memset(&buffer, 0, sizeof(buffer));\
    \}\
    printf("File sent!");\
    \
    close(serv);\
    return 0;\
\}\
\
\\end\{minted\}\
\\noindent%\
From the server, we type gcc server.c -o server. Then type ./server.\
\\noindent%\
We have implemented the server side:\
\\begin\{minted\}\{c\}\
#include <stdio.h>\
#include <stdlib.h>\
#include <string.h>\
#include <sys/types.h>\
#include <sys/socket.h>\
#include <netdb.h>\
#include <unistd.h>\
\
int main() \{\
    int ss, cli, pid;\
    struct sockaddr_in ad;\
    char s[100];\
    socklen_t ad_length = sizeof(ad);\
\
    // create the socket\
    ss = socket(AF_INET, SOCK_STREAM, 0);\
\
    // bind the socket to port 12345\
    memset(&ad, 0, sizeof(ad));\
    ad.sin_family = AF_INET;\
    ad.sin_addr.s_addr = INADDR_ANY;\
    ad.sin_port = htons(12345);\
    bind(ss, (struct sockaddr*)&ad, ad_length);\
\
    // then listen\
    listen(ss, 0);\
\
    while (1) \{\
        // an incoming connection\
        cli = accept(ss, (struct sockaddr*)&ad, &ad_length);\
\
        pid = fork();\
        if (pid == 0) \{\
            // I'm the son, I'll serve this client\
            printf("client connected\\n");\
            \
            FILE* file;\
            file = fopen("recieve_file.txt", "w");\
            if (file == NULL) \{\
            	printf("Cannot open file");\
            \} else \{\
            	printf("Start writing file\\n");\
            \}\
            \
            char buffer[1024];\
            while (1) \{\
                int i = recv(cli, buffer, sizeof(buffer), 0);\
                if (i <= 0) \{\
                	break;\
                \}\
                fprintf(file, "%s", buffer);\
                memset(&buffer, 0, sizeof(buffer));\
            \}\
            printf("File received!");\
            \
            return 0;\
        \} else \{\
            // I'm the father, continue the loop to accept more clients\
            continue;\
        \}\
    \}\
    // disconnect\
    close(cli);\
    close(ss);\
\
\}\
\\end\{minted\}\
\\noindent%\
Here is the result:\
\\begin\{figure\}[H]\
    \\centering\
    \\includegraphics[scale=0.5]\{client.PNG\}\
    \\caption\{Client side\}\
\\end\{figure\}\
\
\\begin\{figure\}[H]\
    \\centering\
    \\includegraphics[scale=0.5]\{server.PNG\}\
    \\caption\{Server side\}\
\\end\{figure\}\
\\subsection\{Contribution\}\
\\noindent%\
\\begin\{table\}[ht!]\
  \\begin\{center\}\
    \\label\{tab:table1\}\
    \\begin\{tabular\}\{l|l\}\
      \\textbf\{Member\} & \\textbf\{Contribution\}\\\\\
      \\hline\
      Nguyen Quang Vinh & Client code\\\\\
      Nguyen Tran Nguyen & Server code\\\\\
      Mai Xuan Hieu & Design Protocol\\\\\
      Nguyen Anh Quan & Design Architecture\\\\\
      Nguyen Tuong Quynh & Report\\\\\
    \\end\{tabular\}\
    \\caption\{Contribution Table\}\
  \\end\{center\}\
\\end\{table\}\
\
\
\\end\{document\}}